\chapter*{Foreword}
\thispagestyle{empty}

\vspace{-2cm}
\vfill

\begin{center}
\begin{minipage}{15cm}
Generally speaking, studying drone's security is more and more exciting as this lucrative market deeply developed during the last decade, opening the way to various commercial usages (for taking selfie photos, delivering goods or yet monitoring areas) but also as weapons. The literature in this field while already broad, as far as we know, does not highlight any tried and tested methodology or does not even provide any extensive open source framework for efficiently assessing drones.

\vspace{-1.2cm}
\epigraph{``Start small, then grow your scope."}{\normalfont \textsc{James Tarala}, SANS Instructor}
\vspace{-1.2cm}

At a first glance, for research purpose, it is then necessary to start from the current state-of-the-art and to test classical techniques on simple drones before addressing more sophisticated ones. Also, we want to make something we can reuse and extend to more complicated use cases. Then building a framework we can apply on new cases emerges as an obvious part of our research. Consequently, we choose a few light commercial drones, limiting our scope to some low-cost and easy-to-address items before being able to grow it to higher technologies. This is the foundation of this work.

\epigraph{``When security is all that matters, security is often overlooked."}{Ir \textsc{Alexandre D'Hondt}, Cybersecurity Expert}
\vspace{-1.2cm}
\hyphenation{competitive}
This is a key point in this work, especially when it comes to Internet of Things and more specifically light commercial drones. Indeed, it is not rare to find such devices with few or even no security implemented, especially when it comes to capturing a very competitive market. The functionality then becomes all that matters, leaving commercialized solutions with security holes that can sometimes be exploited from a smartphone with a free application in a few minutes. This dissertation will show you why exactly.

\vspace{1cm}
\begin{flushright}
\begin{minipage}{4cm}
\raggedleft
A. D'Hondt, Ir
\end{minipage}
\end{flushright}

\end{minipage}
\end{center}

\vfill
