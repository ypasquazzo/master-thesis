\chapter*{Foreword}
\thispagestyle{empty}

\vspace{-2cm}
\vfill

\begin{center}
\begin{minipage}{15cm}
Generally speaking, studying drone's security is more and more exciting as this lucrative market deeply developed during the last decade, opening the way to various commercial usages (taking photos, delivering goods, monitoring areas, etc.) but also to military usages such as weapons. However, the literature in this field, as far as we know, does not highlight any tried and tested methodology nor does provide any extensive open-source framework for efficiently assessing drones.

\vspace{-1.2cm}
\epigraph{``Start small, then grow your scope."}{\normalfont \textsc{James Tarala}, SANS Instructor}
\vspace{-1.2cm}

At a first glance, for research purpose, it is necessary to start from the current state of the art and to test classical techniques on simple drones before addressing more sophisticated ones. Also, we want to make a framework that can be reused and extended to new and more complex cases. Consequently, we choose a few light commercial drones, limiting our scope to some low-cost and easy-to-address items before being able to grow it to higher technologies.

\epigraph{``When security is all that matters, \newline security is often overlooked."}{Ir \textsc{Alexandre D'Hondt}, Cybersecurity Expert}
\vspace{-1.2cm}
\hyphenation{competitive}
This quotation relates to a key point in this work, especially when it comes to Internet of Things and more specifically light commercial drones. Indeed, for such competitive market, it is not rare to find drones with few or even no security implemented. The functionality then becomes all that matters, leaving commercialized solutions with security holes that can sometimes be exploited from a smartphone with a free application in a few minutes. This dissertation will show you why exactly.

\vspace{1cm}
\begin{flushright}
\begin{minipage}{4cm}
\raggedleft
A. D'Hondt, Ir
\end{minipage}
\end{flushright}

\end{minipage}
\end{center}

\vfill
