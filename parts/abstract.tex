\chapter*{Abstract}
\thispagestyle{empty}

\vspace{-3cm}
\vfill

\begin{center}
\begin{minipage}{15cm}
Nowadays, information security takes an increasing place in the world of Information Technologies. We work using software products provided by various companies all around the world, often assuming that these products are safe for use. Because of the rapid proliferation of Internet of Things in consumer and enterprise spheres over the last decade, the interest of malicious individuals consequently arises. Yet, security often comes second to functionality, without further thought of the possible consequences. 
\newline

To cope with this need in computer security, a new profession quickly arose : Penetration Tester. This job consists in testing the security of information systems. In order to know where you might have a breach and patch it before it is exploited, the most efficient solution is to test the limits of your system yourself.
\newline

In particular, the drones are some of the most proliferating Internet of Things, being of interest for everyone, from children to seniors, from playing to working purposes. They range from cheap radio-guided toys, to very expansive GPS controlled tools, with of course the popular smartphone operated flying cameras. This allows a nefarious operator to snoop places that are not supposed to be public, being the privacy of the neighbor's home or a government facility. But the drones aren't just hazardous because of the sensitive information they can reveal, they can also be used as weapons.
\newline

In this scope, our work aims to address the security of light commercial drones and to test the possibility of breaking into these devices from a software point of view. To do so, we opted for an approach based on penetration testing in order to reveal the breaches, and then exploit those breaches to our advantage. We had no prior knowledge on the devices we tested, nor on the drone-related software development in general. We mostly relied on general IT background, as well as various tools developed by security professionals.
\newline

In this master thesis, we propose to apply some common hacking techniques and develop some exploit scripts to automate them. Ultimately, our contribution is to turn these scripts into modules for a brand new modular and extensible penetration testing framework in order to achieve penetration tests on drones in a structured and repeatable way. The framework is designed in the form of a command-line interface, mimicking the very popular toolkit called Metasploit. Being open source and modular, the framework hopes to catch the interest of drone security enthusiasts that are willing to contribute and develop exploits for more drone models in the future. 
\end{minipage}
\end{center}

\vfill
